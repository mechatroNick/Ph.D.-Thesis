\begin{abstract}


Needle-free jet injection allows delivery of liquid drugs through the skin in the form of a narrow fluid jet traveling at high speed, minimizing the risk of accidents. The use of a controllable actuator to drive this process has many advantages, but the voice coil actuators previously used for this purpose are too large and heavy for practical use with common injection volumes (1 mL). Linear synchronous motors, on the other hand, promise significant mass reduction for future portable jet injection systems.


In this thesis, the requirements on linear synchronous motors were examined and tailored to selecting a linear synchronous motor design capable of delivering 1 mL jet injections in the form of a clinically appropriate injectors. The types of linear synchronous motors included Permanent Magnet Linear Synchronous Motor (PMLSM), Linear Flux Switching
Motor (LFSM), and Linear Transverse Flux Motor (LTFM).


Two modelling approaches for linear synchronous motors were developed: A semi-analytical solution for Harmonic Modeling for PMLSM; and a Response Surface Modelling, powered by Artificial Neural Network for PMLSM, LFSM, and LTFM. Both Harmonic Modeling and Response Surface Modelling were found to have very similar modeling results. All three types of motors were optimized to the same set of requirements previously engineered with minor adaptations. PMLSM was found to be the best performing type of motor in this case study.


A final prototype motor design was determined using an optimization scheme for finding the lowest motor mass at a fixed power dissipation, and an automated routine for estimating the cogging force using finite-element analysis. A prototype motor was constructed, with a nominal mass of 322 g, a stroke of 80\,\,mm, and a target operating power of 1.2 kW; experimental data show that the motor constant is within 10\% of the target, and that the cogging force is in close agreement with the model. Test ejection of water into a force sensor and porcine tissue verified that the motor is fit for delivering 1 mL needle-free injections. 


The design methodology explained here shows the benefits to integrated design optimization of both the actuator and the load, particularly in systems that drive fluid pressure loads, and also opens the door to controllable injector designs for larger volumes. 


\end{abstract}
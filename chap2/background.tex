\chapter{Background}                                \label{Chapter:background}


% ===================================================================================================
% === NEW SECTION === NEW SECTION === NEW SECTION === NEW SECTION === NEW SECTION === NEW SECTION ===
% ===================================================================================================
\section{Transdermal drug delivery}                 \label{Chapter:background/transdermal drug delivery}
    % \subsection{Problems}                           \label{Chapter:background/transdermal drug delivery/problems}
    % \subsection{Alternative methods}                \label{Chapter:background/transdermal drug delivery/alternative methods}

    Transdermal drug delivery is a method of delivery for pharmaceutical solutions that are unsuitable for ingestion or cannot penetrate through the skin. Many vaccines and insulin are delivered by this approach\,\cite{sadrzadeh2007}. While being efficient and precise, transdermal drug delivery via hypodermic needle is time-consuming, labor intensive, and hazardous. The procedure requires the operator to attach a hollow needle to a syringe, then extract the drug, eliminate air bubbles in the syringe and sterilize the applied area thoroughly. Once prepared, the operator can slowly inject a high volume of drug. During disposal, needle stick is a safety hazard. Needle-stick injuries hold a high risk of transmitting contagious diseases such as HIV, HBV, and HCV. Percutaneous sharps injuries affected millions of individuals across the world\,\cite{pruss2005}. 
    
    Today, needle sticks and sharp objects still represent a significant challenge in creating a safe environment for professional health care practitioners. Pain\,\cite{schneider1994} and needle-phobia\,\cite{hamilton2005,Nir2003} are other motivations to develop and popularize alternative strategies.
    
    Some distinct transdermal drug delivery methods were invented to tackle issues of needle injection. They include iontophoresis\,\cite{dhote2012}, sonophoresi\,\cite{bommanan1992}, permeation enhancement by chemicals\,\cite{karande2006}, micro-needles on patch\,\cite{cormier2004}, and jet injection\,\cite{taberner2006}. 

% ===================================================================================================
% === NEW SECTION === NEW SECTION === NEW SECTION === NEW SECTION === NEW SECTION === NEW SECTION ===
% ===================================================================================================
\section{Needle-free Jet Injection}                 \label{Chapter:background/needle-free jet injection}
    
    
    % -----------------------------------------------------------------------------------
    % --- NEW SUB SECTION --- NEW SUB SECTION --- NEW SUB SECTION --- NEW SUB SECTION --- 
    % -----------------------------------------------------------------------------------
    \subsection{How it works}                       \label{Chapter:background/needle-free jet injection/how it works}
    
        \ac{NFJI}, commonly called “hypo-spray” in science fiction, was patented in 1960\,\cite{ismach1962}. Figure\,\ref{fig:chapter/background/explain needle free/original injector} shows the original \acs{NFJI} device, “Ped-O-Jet” in use for the purpose of mass vaccination\,\cite{DictionnairesetEncyclopediessurAcademic}. This device was invented  upon realizing that pressurized fluid can penetrate human skins. Fluid streams of appropriate diameter and velocity can produce sufficient pressure to breach through the skin layers up to a particular desired depth. 
        
        \begin{figure*}[!ht]
            \centering
            \subfloat[Ped-O-Jet in use]{
                \includegraphics[width=0.35\textwidth]{chap2/images/original_injector.png}
                \label{fig:chapter/background/explain needle free/original injector}
            }
            \qquad
            \subfloat[Needle injection versus Needle-free injection]{
                \includegraphics[width=0.45\textwidth]{chap2/images/needle_vs_needle_free.png}
                \label{fig:chapter/background/explain needle free/needle vs no needle}
            }
            \caption{
                Two different style of optimization objective function evaluation.
            }   \label{fig:chapter/background/explain needle free}
        \end{figure*}
        
        A subcutaneous or intramuscular jet injection could be realized by forcing a fluid jet of $\mathrm{76-360\,\mu m}$  diameter to penetrate the human skin at a speed faster than $\mathrm{100 m/s}$\,\cite{mitragotri2006,Hogan2006}. Figure\,\ref{fig:chapter/background/explain needle free/needle vs no needle} shows that an ideal jet injection would deliver the drug in a similar manner to which of a hypodermic needle injection\,\cite{InsuJet2013}.
        
        % Even though this method adopts the use of needle-free apparatus, biological material can still be unwillingly transferred if there are no appropriate decontamination schemes. Concerns have been raised in the literature about potential transmission of blood-borne infections by multiple-use \acs{NFJI} since the early 1970s\,\cite{kremer1970, weintraub1988}. Figure\,\ref{fig:chapter/background/injection mechanism} illustrates three possible mechanisms of how blood contamination can take place\,\cite{hoffman2001}. 
        
        % \begin{figure*}[!ht]
    
        %     \centering
        %     \subfloat[]{
        %         \includegraphics[width=0.29\textwidth]{chap2/images/jet_injection_mechanism_1.png}
        %         \label{fig:chapter/background/injection mechanism/1}
        %     }
        %     \subfloat[]{
        %         \includegraphics[width=0.25\textwidth]{chap2/images/jet_injection_mechanism_2.png}
        %         \label{fig:chapter/background/injection mechanism/2}
        %     }
        %     \subfloat[]{
        %         \includegraphics[width=0.27\textwidth]{chap2/images/jet_injection_mechanism_3.png}
        %         \label{fig:chapter/background/injection mechanism/3}
        %     }
        %     \caption{
        %         3 possible mechanisms of blood contamination in ‘mass campaign jet injectors’. 
        %     }   \label{fig:chapter/background/injection mechanism}
        % \end{figure*}    
        
        % Scenario in Figure\,\ref{fig:chapter/background/injection mechanism/1} shows that tissues return fluid into the reservoir as injection pressure diminishes. Liquid flows out to the tip of the injector as the injector’s tip is removed from the applied area as illustrated in Figure\,\ref{fig:chapter/background/injection mechanism/2}. Scenario in Figure\,\ref{fig:chapter/background/injection mechanism/3} displays a ‘splash back’ of jet stream during injection. Uninterrupted, continuous, reuse of \acs{NFJI} has historically caused instances of mass HBV spread\,\cite{canter1990}. For prevention of future mass infection, ‘mass campaign jet injectors’ were disapproved for human use by the World Health Organization\,\cite{who2005}. Nowadays, reusable \acsp{NFJI} for human use must accommodate replaceable syringes. 
    
    
    % -----------------------------------------------------------------------------------
    % --- NEW SUB SECTION --- NEW SUB SECTION --- NEW SUB SECTION --- NEW SUB SECTION --- 
    % -----------------------------------------------------------------------------------
    \subsection{Underlying mechanics}               \label{Chapter:background/needle-free jet injection/underlying mechanics}
    
        \begin{figure*}[!ht]
            \centering
            \subfloat[]{
                \includegraphics[width=0.28\textwidth]{chap2/images/jet_injection_mechanics_1.png}
                \label{fig:chapter/background/underlying mechanics/1}
            }
            \quad
            \subfloat[]{
                \includegraphics[width=0.28\textwidth]{chap2/images/jet_injection_mechanics_2.png}
                \label{fig:chapter/background/underlying mechanics/2}
            }
            \quad
            \subfloat[]{
                \includegraphics[width=0.3\textwidth]{chap2/images/jet_injection_mechanics_3.png}
                \label{fig:chapter/background/underlying mechanics/3}
            }
            \caption{
                (a) The general shape of jet penetration into polyacrylamide gel; (b) Side view of the same gel sample; (c) The transverse slice of the gel shows the presence of the cylindrical channel.
            }   \label{fig:chapter/background/underlying mechanics}
        \end{figure*}
    
        To explore mechanics of NFJI into the skin, Schramm-Baxtex et al.\,\cite{Schramm-Baxter2004b} conducted an experiment where a high-speed camera monitored a spring-driven jet injection into polyacrylamide gel as a test bed as shown in Figure\,\ref{fig:chapter/background/underlying mechanics}. Motion analysis demonstrated the presence of three distinct jet injection phases: erosion, stagnation, and dispersion. The erosion phase is the period when jet penetrates in the form of a cylindrical channel. The brief reduction of upfront kinetic energy causes the fluid to start building up at a certain depth, described as the stagnation phase. The dispersion phase is where a drug volume infiltrates the cracks propagated within the gel by fluid pressure during previous phases. The maximum depth was controlled by altering the jet velocity at erosion, and likewise, the amount of drug delivered is determined in by the jet speed at the dispersion phase\,\cite{Stachowiak2009}.
        
        
        \begin{figure*}[!ht]
            \centering
            \subfloat[]{
                \includegraphics[width=0.515\textwidth]{chap2/images/1_speed_injection.png}
                \label{fig:chapter/background/2 speeds mechanics/just 1 speed}
            }
            \qquad
            \subfloat[]{
                \includegraphics[width=0.20\textwidth]{chap2/images/2_sppeds_injection.png}
                \label{fig:chapter/background/2 speeds mechanics/2 speeds}
            }
            \caption{
                Need for temporal control of jet velocity in needle-free transdermal drug delivery: (a) constant low-velocity delivery (left), medium-velocity delivery (center), and high-velocity delivery (right); (b)Desired drug delivery showing sufficient penetration depth and minimal pooling. Dotted line represents the desired depth of injection.
            }   \label{fig:chapter/background/2 speeds mechanics}
        \end{figure*}
        
        
        Difficulty in controlling depth injection and ‘splash back’ was the result of using a single jet velocity\,\cite{schramm2002}. Figure\,\ref{fig:chapter/background/2 speeds mechanics/just 1 speed} explains the need for temporal control of jet velocity in needle-free transdermal drug delivery: (a) constant low-velocity delivery (left), medium-velocity delivery (center), and high-velocity delivery (right); (b)Desired drug delivery showing sufficient penetration depth and minimal pooling. Dotted line represents the desired depth of injection[35]. At constant low pressure, the jet may not penetrate into the skin (left), while constant, medium-pressure jet velocity may breach the skin with minimal ‘splash back’, yet reach an insufficient depth (middle).A high-pressure stream may reach the desired depth, but fluid pooling cannot be avoided (right). With the use of an initial high jet pressure followed by a lower holding pressure, it was predicted that ’splash back’ would be reduced while the jet disperses at the correct depth\,\cite{wendell2006}. This concept is portrayed in Figure\,\ref{fig:chapter/background/2 speeds mechanics/2 speeds}. Taberner et al.\,\cite{taberner2012} demonstrated that the use of a high degree of fluid stream velocity control can deploy this strategy reliably.
    
    
    % -----------------------------------------------------------------------------------
    % --- NEW SUB SECTION --- NEW SUB SECTION --- NEW SUB SECTION --- NEW SUB SECTION --- 
    % -----------------------------------------------------------------------------------
    \subsection{Commercially available options}     \label{Chapter:background/needle-free jet injection/commercially available options}
        
        Commercially available \acsp{NFJI} including BIOJECT Zeajet\footnote{http://www.bioject.com/products/zetajet-info} , INJEX 30\footnote{https://www.injex.com.au/injex/injex} , PHARMAJET Stratis\footnote{http://pharmajet.com/fda-approved-needleless-flu-shot} , COMFORT-IN\footnote{http://www.comfort-in.com/diabetes.html}  are spring-powered, small, portable, and handheld devices. They are capable of delivering either vaccination or insulin in a small dose, with volume limited to $\mathrm{300\,\mu L}$. Figure\,\ref{fig:chapter/background/pharma jet stratis} illustrates the appearance, construction, and mechanism of the spring-loaded PHARMAJET Stratis NFJI as an example. These devices come in various shapes, sizes, injection volumes, and target skin layers. However, their primary structure consists of three main components:
        \begin{itemize}
            \item Energy storage - compressed gas, spring coil or explosives\,\cite{taberner2012}, which are often accompanied by a recharging device,
            \item Piston - actuator with various size and triggering method to accomplish desired pressure profile as well as injection volume specification,
            \item Replaceable syringe - single use container with orifice hole diameter of under $\mathrm{1\,mm}$.
        \end{itemize}
        
        \begin{figure*}[!ht]
            \centering
            \subfloat[\acs{NFJI} device and replaceable syringe]{
                \includegraphics[width=0.40\textwidth]{chap2/images/pharma_jet_stratis.png}
                \label{fig:chapter/background/pharma jet stratis/full view}
            }
            \qquad
            \subfloat[Spring loaded mechanism of the device]{
                \includegraphics[width=0.45\textwidth]{chap2/images/pharma_jet_stratis_cut_view.png}
                \label{fig:chapter/background/pharma jet stratis/cut view}
            }
            \caption{
                SUNFJI Pharma Jet Stratis™ from PharmaJet™
            }   \label{fig:chapter/background/pharma jet stratis}
        \end{figure*}
        
        Once triggered, stored energy produces varying pressure on the injection cylinder until the piston reaches the end of its track. Different devices utilize different recoil or trigger mechanisms, but no system is capable of monitoring and measuring injection performance. Instead of concentrated fluid delivery in a narrow stream down the tissue, these devices tend to spread the drug content in a cone shape\,\cite{baxtex2005}. Figure\,\ref{fig:chapter/background/jet injection effectiveness/mechanical devices pressure curve} shows the sharp peak and oscillating nature of the pressure produced by injection stream of a BIOJECT Vitajet 3™ spring powered \acsp{NFJI}\,\cite{schramm2002}. Schneider et al.\,\cite{schneider1994} reported that without stroke velocity and position control, the sharp fluid pressure profile peak could cause pain, bruises, bleeding, and blisters. Their studies also showed evidence that the same spring powered NFJI device exhibits inconsistent performance across different skin types and conditions. 
        
        Despite being a safer alternative, mechanically powered \acsp{NFJI} require manual recharging. Thus, the average cycle time was recorded to be no faster than that of a conventional hypodermic needle injection\,\cite{PharmaJet2011}. More advanced \acsp{NFJI} like LECTRAJET  have shown the capability of automatic spring recoil to reduce reload time. Lack of pressure control, low injection volume (typically $0.05$ to $0.3\,\mathrm{mL}$), long cycle time, poor repeatability and reproducibility are drawbacks of commercially available mechanically powered \acsp{NFJI}.
        
        \begin{figure*}[!ht]
            \centering
            \subfloat[Typical pressure curve for a $152\,\mathrm{\mu m}$ diameter nozzle from injector Vitajet\,3™]{
                \includegraphics[width=0.43\textwidth]{chap2/images/jet_injection_pressure_curve.png}
                \label{fig:chapter/background/jet injection effectiveness/mechanical devices pressure curve}
            }
            \qquad
            \subfloat[Delivery of mannitol by jet injection into human (A), porcine abdominal (B), and porcine dorsal skin (C) using the same device at $177\,\mathrm{m/s}$. There is a significant difference between each type of skin tested ($p=0.001$)]{
                \includegraphics[width=0.45\textwidth]{chap2/images/jet_injection_delivery_study.png}
                \label{fig:chapter/background/jet injection effectiveness/delivery statistic}
            }
            \caption{
                Results of different jet injection study on mechanically powered \acs{NFJI} devices demonstrating rough pressure profile and poor adaptability.
            }   \label{fig:chapter/background/jet injection effectiveness}
        \end{figure*}


    % -----------------------------------------------------------------------------------
    % --- NEW SUB SECTION --- NEW SUB SECTION --- NEW SUB SECTION --- NEW SUB SECTION --- 
    % -----------------------------------------------------------------------------------
    \subsection{Controllable \acs{NFJI} devices}    \label{Chapter:background/needle-free jet injection/Controllable NFJI}
    
        Recent achievements in high power density actuators enabled successful prototypes of electronic controlled \acsp{NFJI} which were capable of real-time jet velocity control. Hemond et al. [24] suggests four advantages of electronics control over traditional needle injection and mechanically powered jet injection:
        
        \begin{itemize}
            \item Controllable - capable of maintaining high pressure without the need of trading-off for injection volume,
            \item Reliable - capable of producing consistent jet pressure on a broad range of skin properties and different injection types,
            \item Versatile - automatic reload under closed loop control or injection volume control,
            \item Measurable - convenient in monitoring progress to evaluate the performance of injection.
        \end{itemize}
        
        This class of device performs injections by utilizing different electrically powered actuators: dynamically controlled piezo-electric actuators\,\cite{Stachowiak2009}, laser pulsed microjets\,\cite{tawaga2013, park2012} and Lorentz force voice coil motors\,\cite{taberner2006,hemond2006}. Piezo-electric and laser pulse methods are limited to sub $\mathrm{\mu L}$ injections. Scaling of Piezo-electric technology to   range in injection volume appear ambitious. Cumbersome ‘off-the-shelf’ voice coil actuators which will be explored in \ref{Chapter:background/voice coil motors for NFJI} are not yet suitable to become handheld.

% ===================================================================================================
% === NEW SECTION === NEW SECTION === NEW SECTION === NEW SECTION === NEW SECTION === NEW SECTION ===
% ===================================================================================================
\section{Voice coil motors for \acs{NFJI}}          \label{Chapter:background/voice coil motors for NFJI}
    
    
    % -----------------------------------------------------------------------------------
    % --- NEW SUB SECTION --- NEW SUB SECTION --- NEW SUB SECTION --- NEW SUB SECTION --- 
    % -----------------------------------------------------------------------------------
    \subsection{Principle of operation}             \label{Chapter:background/voice coil motors for NFJI/principle}


    The \ac{VCM} is a type of Lorentz-force actuator. \ac{VCM} can either have the magnet or the coil as moving part. Figure\,\ref{fig:chapter/background/vcm cut view} presents the construction of moving coil configuration\,\cite{taberner2006}. Here, a Nd-Fe-B magnet is used to generate the magnetic field. The magnet is placed between a steel top plate and the steel casing to direct the flow of magnetic flux in a contained loop. In between the top plate and steel case is a region of air gap that allows of copper winding to slide up or down. Linear force $F$ is proportional to strength of the magnetic field $B$ through wire, and the electric current per unit area $J$ passing through the coil:
    
    
    \begin{equation}
        F=J\times B
        \label{eq:force produce via field and current}
    \end{equation}

    The polarity of the current also dictates which direction the coil will move. It is clear that more force can be generated by either strengthening the magnetic field or increasing the current through the coil. Adjusting other parameters such as coil geometry, coil thickness, magnet materials, and aspect ratio can alter the motor force constant and stroke length.
    
    
    \begin{figure}[!ht]
      \centering
      \includegraphics[width=0.5\textwidth]{chap2/images/vcm_cut_view.png}
      \caption{Construction of a voice coil motor with moving coil.}
      \label{fig:chapter/background/vcm cut view}
    \end{figure}
    
    
    % -----------------------------------------------------------------------------------
    % --- NEW SUB SECTION --- NEW SUB SECTION --- NEW SUB SECTION --- NEW SUB SECTION --- 
    % -----------------------------------------------------------------------------------
    \subsection{Application to \acs{NFJI}}          \label{Chapter:background/voice coil motors for NFJI/application}
    
    
    Recently, collaboration between the \ac{MIT} and \ac{ABI} Bioinstrumentation Labs has created two prototypes jet injectors actuated by customized Lorentz-force \ac{VCM}\,\cite{taberner2006,ruddy2014} shown in Figure\,\ref{fig:chapter/background/vcm injectors}. With the use of real-time feedback control, these systems have demonstrated highly controllable and repeatable injections against a broad range of different test tissues, depths, and dosage volumes\,\cite{taberner2012}. With the successful prototypes, the vision was to advance the prototypes into portable and high volume injector devices.


    \begin{figure*}[!ht]
        \centering
        \subfloat[]{
            \includegraphics[width=0.5\textwidth]{chap2/images/vcm_taberner2006.png}
            \label{fig:chapter/background/vcm injectors/taberner2006}
        }
        \qquad
        \subfloat[]{
            \includegraphics[width=0.28\textwidth]{chap2/images/vcm_ruddy2014.png}
            \label{fig:chapter/background/vcm injectors/ruddy2014}
        }
        \caption{
            Lorentz-force \acs{VCM} actuated \ac{NFJI}: Handheld injector and cRIO controller system (a); Hand-held injector and charged capacitor amplifier/controller system (b).
        }   \label{fig:chapter/background/vcm injectors}
    \end{figure*}
    
    
    \begin{figure}[!ht]
      \centering
      \includegraphics[width=0.6\textwidth]{chap2/images/vcm_for_nfji.png}
      \caption{A basic schematic of a voice coil actuated \acs{NFJI}. $L$ and $D$ are the cylinder’s working length (stroke length) and diameter, respectively.}
      \label{fig:chapter/background/vcm for nfji}
    \end{figure}
    
    
    Figure\,\ref{fig:chapter/background/vcm for nfji} shows a basic schematic of a voice coil actuated \acs{NFJI}. $L$ and $D$ are the cylinder’s working length (stroke length) and diameter, respectively\,\cite{ruddy2014}. In this device, injection volume can be enlarged by either up-scaling the ampoule diameter $D$, or the working stroke length $L$, or both. A jet injector with an ampoule diameter $D$ requires an actuation force $F$ to be provided by the motor. For constant peak jet velocity $v_{jet}$ and fluid density $\rho$:
    
    
    \begin{equation}
        F=\frac{\pi}{8}\rho {v_{jet}}^2 D^2
        \label{eq:force produce relationship in motor powered NFJI}
    \end{equation}
    
    
    If syringe’s diameter\,$D$ is chosen to scale with injection volume, electric power\,$P$ needs to scale with the square of $D$’s scaling ratio due to the relationship:
    
    
    \begin{equation}
        P=F v_{piston}
        \label{eq:power required for F and v_piston}
    \end{equation}
    
    
    where $v_{piston}$ is the speed of the moving piston. If syringe’s diameter  is chosen to scale with injection volume, electric power\,$P$ needs to scale with the square of $D$’s scaling ratio. Given that the jet velocity follows the varying velocity strategy explained in Section\,\ref{Chapter:background/needle-free jet injection/underlying mechanics}, the motor needs to achieve the peak jet velocity required for the brief erosion phase, then the lower jet velocity speed for dispersion phase. For only for a very short period, the device is required to produce much more power than its average power drawn. It is not common to find energy storage and powering devices at power magnitude of many kilowatts, provided in a portable fashion.
    
    
    % -----------------------------------------------------------------------------------
    % --- NEW SUB SECTION --- NEW SUB SECTION --- NEW SUB SECTION --- NEW SUB SECTION --- 
    % -----------------------------------------------------------------------------------
    \subsection{Scaling properties and limitations} \label{Chapter:background/voice coil motors for NFJI/scaling and limitation}
    
    
    The following key relationship relates the power dissipation on the motor winding $P$, density of fluid to be deliver $\rho$, ampoule volume $V$, jet velocity $v_{jet}$, motor constant $K_m$, and distance travelled by the piston stroke $L$ \cite{Williams2012}:
    
    
    \begin{equation}
        P=\frac{\rho^2 V^2 {v_{jet}}^4}{4 K_m L^2}
        \label{eq:power required for F,V,v_jet,K_m, and L}
    \end{equation}
    
    
    To compare the force exerted over power consumed in different motors, the motor constant $K_m$ measured in $\mathrm{N/\sqrt{W}}$  is typically used. Note that this relationship is true of all direct drive linear motors, which apply force directly to single injection ampoule without any mechanical coupling transmission.
    
    
    Utilizing a general scaling magnetic and thermal model made for steady-state force production of \ac{PMLSM} constructed by Ruddy et al.\,\cite{Ruddy2011DesignMotors}, an optimal \acs{VCM} hand-piece was built to deliver $\mathrm{300\,\mu L}$ of the drug over a $\mathrm{50\,ms}$ period\,\cite{taberner2006}. The peak power required for this procedure was as high as 10 kW to produce peak force of $\mathrm{300\,N}$ of over the stroke length of $\mathrm{35\,mm}$. Due to the extremely short duration of the force profile, the heat generated is far from enough to cause permanent demagnetization in the permanent magnet array; thus, the motor design neglect heat transfer. In the same body of work, the authors pointed out that scaling laws of voice coil actuator fit for needle free injection means that the power $P$ required and motor mass $M$ both grow faster than injection volume $V$:
    
    
    \begin{equation}
        M \propto V^{6/5}
        \label{eq:scaling property of VCM}
    \end{equation}


    While the potential of using \acsp{VCM} in portable \acs{NFJI} devices has been proven, this type of motor is power inefficient. Without breakthrough improvements, \acsp{VCM} pose a considerable restriction on the portability of the electronics control system, as well as the size of the injector handpiece motor. Based on the model presented in that work, a \acs{VCM} with mass of over $\mathrm{1\,kg}$ is required to deliver $\mathrm{1\,mL}$. Thus, it was impractical to use a direct-drive linear \acsp{VCM} to deliver regular livestock injections, where the volume can be as high to $\mathrm{10\,mL}$. 

% ===================================================================================================
% === NEW SECTION === NEW SECTION === NEW SECTION === NEW SECTION === NEW SECTION === NEW SECTION ===
% ===================================================================================================
\section{Linear synchronous motors for NFJI}        \label{Chapter:background/linear synchronous motors for NFJI}

    
    \acp{LSDDM} are actuators that can drive a motion load without an intermediate mechanism such as gears, screws or crank shafts\,\cite{JacekF.GierasZbigniewJ.Piech2017LinearSystems}. The absence of auxiliary mechanical adaptation increases their reliability. In high speed application, linear direct-drive synchronous motors can have higher efficiency and a higher dynamic performance compared to their rotary counter part. ‘Synchronous’ in the context of electric machines means that the speed of motion is the same as the speed of traveling magnetic field. The thrust is generated by either the interaction between the traveling magnetic field produced by poly-phase winding or switched DC windings in an array of magnetic excitation components. 
    
    \acsp{LSDDM} are being used increasingly in applications as varied as automated manufacturing\,\cite{Meessen2010Three-dimensionalArray,Overboom2010DesignZ-module}, transportation\,\cite{Gysen2011EfficiencySuspensionb,Cao2012,WenxiangZhao2012DesignApplications}, and power generation\,\cite{Li2011,Baker2019AEnergy}. Although \acsp{LSDDM} have existed for a long time\,\cite{Boldea1997}, to the best of the author’s knowledge, no report has demonstrated that this class of motor can achieve \acs{NFJI} tasks, where a high pulse of force is required from near stall condition. Technically speaking, \acs{VCM} is a type of \acsp{LSDDM} where there are a single-phase winding driven by a direct current and an array of magnetic excitation with length equal one pole period. True linear synchronous motors, powered by alternating current, provides efficient energy to motion conversion, high force density and excellent capability for precise motion control\,\cite{Trumper1994,Levi1973,Budig2000}.
    
    
    % -----------------------------------------------------------------------------------
    % --- NEW SUB SECTION --- NEW SUB SECTION --- NEW SUB SECTION --- NEW SUB SECTION --- 
    % -----------------------------------------------------------------------------------
    \subsection{Classification}                     \label{Chapter:background/linear synchronous motors for NFJI/classification}
    
        
        The most basic motor consists of two principal components\,\cite{JacekF.GierasZbigniewJ.Piech2017LinearSystems}: the part that produces traveling magnetic field via poly-phases copper winding is the armature, and the part that produces magnetic flux or variable reluctance is known as the field excitation system (e.g., permanent magnet or magnetic field created by induction). \acsp{LSDDM} can come in many different topologies\,\cite{Laithwaite1970}:
        
        
        \begin{itemize}
            \item Moving or fixed armature
            \item Flat (planar) or tubular (cylindrical)
            \item Ferromagnetic core (back iron or dual Halbach magnet array\,\cite{Halbach1980}) or air core
            \item Slotted or slot-less core (if using back iron core)
            \item Permanent Magnet excitation or Electromagnetic excitation
            \item Longitudinal flux or Transversal/Transverse flux
            \item Singly salient or Doubly salient
        \end{itemize}
        
        
        Moving or fixed armature simply refers to which part of the motor moves, be it the part containing the armature or part that contain the field excitation system. A moving armature system requires more materials for the fixed secondary, which often increase the demand for costly rare-earth permanent magnet. In contrast, the a moving secondary system often leads to more demanding electronics to control more armature phases. Moving armature configuration is more common seen in linear direct drive motors. 
        
        
        Tubular motor concept can be realized by rolling flat linear machine about their longitudinal axis. Tubular permanent-magnet machines offer the highest efficiency and power over force density and have excellent servo characteristics\,\cite{Eastham1990NovelDisc}. The advantages tubular structure can offer over flat structure includes even magnetic field distribution and ease of modelling. Despite superior performance, tubular machines are only more popular than flat machine in research because they are more difficult to manufacture at scale.
        
        
        Ferromagnetic core, whether it is magnet core or iron core, exist to provide a low magnetic resistance path between the excitation source and the armature. The subset of \acp{PM} synchronous motor with the use of iron core has to deal with cogging force. This force stands for the mutual attraction between PM and iron in the armature. Cogging force is undesirable because it can contribute to a reduction in axial force, vibration and deteriorate control characteristic at low speed\,\cite{Jung1999}. Slotted iron core causes the cogging ripples while limited iron length causes end effect cogging\,\cite{Lim2002}. Cogging force can be reduced by PM width adjustment\,\cite{Bianchi2002DesignMotors}, asymmetric PM arrangement\,\cite{Chung2016,Bianchi2003a,Cai2012}, semi-closed slot for iron core\,\cite{Zhu1997,Bai2015ReducingStructure,Zhu2008}, optimizing iron core geometry\,\cite{Inoue2000,Zhu1997,Wu2008,Zhu2009conf,Zhang2013,Kim2016,Lee2014}. It is better to improve cogging force profile using mechanical design instead of relying on complex control mechanism\,\cite{Jahn1996}. Actuators using permanent magnets are the most researched actuators. This is due to the availability of high energy rare earth permanent magnet materials compact actuators with a high efficiency can be realized.
        
        
        The most important distinction for motors perhaps lies in the arrangement of the flux path. Longitudinal flux means that the magnetic field lines are parallel to the direction of motion. Transversal flux, on the other hand, means that the magnetic field lines are perpendicular to the direction of the motion\,\cite{Laithwaite1975LinearView}. Further definition of Transversal flux motor includes having three-dimentional magnetic flux flow, and there is no coupling between individual armature and secondary phases. Transversal flux motors are especially suitable for low speed and high thrust application\,\cite{Zhao2015,Shin2015}. This topology allows for containing a large number of poles without compromising the space available for winding\,\cite{Laithwaite1971}. The structure of this type of motor results in a complex, coupled magnetic and electric circuit. An increase in the number of poles is roughly proportional to the increase in force density. Transversal flux motor, in both rotary and linear forms, can achieve very high work output\,\cite{Ueda2014SmallCondition,Hsu2011DevelopmentMotor,Wang2016OptimalApplications,Arshad2001,Siatkowski2008}. Longitudinal flux machines may have discrete structural poles, but for transversal flux machines, discretization of poles is compulsory\,\cite{Baoming2009DesignMotor}. This structure offers easier and cheaper design, manufacturing and maintenance. However, the segmented core construction and widely open conductor loop lead to high leakage flux, high winding inductance, and low power factor\,\cite{Harris1997ComparisonMachines,Lu2003ModelingMachine}.  Iron core takes significantly more volume and weight in transversal flux motors, thus, Eddy losses and nonlinear permeability effects must be included to model this type of motor accurately.
        
        
        Both Transversal flux motors or Longitudinal flux when called alone, refers to motors that have flux source (\acs{PM} for instance) separated from the current source, and both sources will not move at the same time as the motors operates. Since the flux source interacts with the current source through an air gap without hard mechanical coupling, we refer those motors as Singly salient motors. On the other hand, motors with magnets and coils on the same moving or stationary side are called Doubly salient machines. Belonging  to  the  the  family of doubly-salient permanent magnet motors \,\cite{Cheng2011}, \acp{LFSM} have high  thrust  density, high  tolerance  to  current  overload, lower use  of  permanent  magnet  material,  and  a generally robust construction. They use a passive secondary that can be made out of steel lamination or \ac{SMC} to reduce Eddy current loss  to the minimum. Being  able  to ignore Eddy currents will play an important role in improving the efficiency of design simulation. The amount of \acs{PM} in a \acs{LFSM} in long stroke applications can be significantly less than that used by a \acs{PMLSM}\,\cite{Aleksandrov2018DesignTracks}. All these advantages are important in bringing a  prototype into manufacturing at scale to reach high reliability. In hindsight, air is a very poor magnetic conductor, thus, machines like \acsp{LFSM} with very small flux path resistance should vastly improve motors output. However, since the \acs{LFSM} is still a relatively unexplored topology, the literature has opposing opinions about whether \acsp{LFSM} outperform \acsp{PMLSM}\,\cite{Aleksandrov2018DesignTracks,Wang2008ComparativeSuspension}.
    
    
    % -----------------------------------------------------------------------------------
    % --- NEW SUB SECTION --- NEW SUB SECTION --- NEW SUB SECTION --- NEW SUB SECTION --- 
    % -----------------------------------------------------------------------------------
    \subsection{Advantages and disadvantages}   \label{Chapter:background/linear synchronous motors for NFJI/advantages and disadvantages}
        
        
        One of the disadvantages of synchronous motors is that once the motion and rotating magnetic fields were out synchronism, the motors would not move. A way to work around this problem is to incorporate absolute position sensors or Hall sensors into the control algorithm of the motor. In some cases where the armature requires a large starting torque, the motor needs cleverly fit a ‘Squirrel Cage’\,\cite{Marcic2008} at poles tips because the motor is inherently unable to start itself. Squirrel Cage is a popular self-start mechanism, often seen in rotary synchronous motors to allow for asynchronous starting and catching up with synchronism. Even with the help of methods to speed up to synchronism, the armature can always lag if the power supplied is insufficient to cope with the load.
        
        
        
        Compare to \acsp{VCM}, \acsp{PMLSM} and \acsp{LFSM} in general require far less power to operate. Thus, the same portable power supply built for powering the \acsp{VCM} is also capable of driving the other \acsp{PMLSM} and \acsp{LFSM}, provided that the technology used has room to expand for multiphase control.
        
        
    % -----------------------------------------------------------------------------------
    % --- NEW SUB SECTION --- NEW SUB SECTION --- NEW SUB SECTION --- NEW SUB SECTION --- 
    % -----------------------------------------------------------------------------------
    \subsection{Summary}                         \label{Chapter:background/linear synchronous motors for NFJI/summary}
        
        
        Previous work at the ABI also conceived a design of a \ac{PMLSM} optimized for \acs{NFJI} application\,\cite{Ruddy2015}. The motor described in that work is a moving armature, tubular, single-sided, back iron cored, slot-less, \acs{PM}, and longitudinal flux linear synchronous motor. The optimized tubular \acs{PMLSM} with a stroke length of $\mathrm{200\,mm}$ and an injection volume of $\mathrm{1\,mL}$ promised a drastic reduction in power requirement compared to \acsp{VCM}. The next step from there is to construct a prototype motor and power amplifier, and incorporate them into a complete \acs{NFJI} system. Such motor design will need to incorporate a creative packaging strategy to maintain the hand-held nature of the \acs{NFJI} device. The challenges of constructing \acs{PMLSM} motors with use the of PM also include assembly of a large number of small and mutually attracted parts\,\cite{Hwang2012,Shin2012}.
        
        
        While the scaling law of Longitudinal flux \acs{PMLSM} requires the stroke length to be long to make a drastic improvement in power efficiency\,\cite{Laithwaite1970}, however, that is not the case in Transversal flux motors. The implication is that the uses of long \acs{NFJI} ampoule stroke and cumbersome packaging mechanism can be avoided altogether if we decided to adopt linear Transversal flux motors. The use of newer types of motor such as \acsp{LFSM} should also be explored because it is possible to save a significant amount of rare-earth permanent magnet and near equivalent performance to \acs{PMLSM}.


% ===================================================================================================
% === NEW SECTION === NEW SECTION === NEW SECTION === NEW SECTION === NEW SECTION === NEW SECTION ===
% ===================================================================================================
\section{Electromagnetic field theory}              \label{Chapter:background/electromagnetic field theory}


    Magnetism is one of the first natural force discovered and harvested by human. The emergence of energy convergence of energy conversion applications and modern electronics have led to a large variety of magnetic materials supplied in all shapes and size. The phenomenon of a material forming a permanent magnet or temporary microscopic dipoles under the influence of magnetic field is known as ferromagnetism. While all materials are magnetic to some extent, only ferromagnetic materials can form strong magnetic flux density $B$ under the influence of external magnetizing field $H$. In this body of work, they are referred as ‘$B$-field’ and ‘$H$-field’. 
    

    % -----------------------------------------------------------------------------------
    % --- NEW SUB SECTION --- NEW SUB SECTION --- NEW SUB SECTION --- NEW SUB SECTION --- 
    % -----------------------------------------------------------------------------------
    \subsection{Quasi-static Maxwell equations}     \label{Chapter:background/electromagnetic field theory/quasi-static maxwell equations}
    
    
        Maxwell equations describe how electric and magnetic field are generated by charges, current, or the result of changes in each other. As a set of \acp{PDE}, they provide a foundation for understanding classical electromagnetism, optics and even electrical circuits. In the case of motors, the electromagnetic fields travel drastically slower than the maximum speed possible for electromagnetic waves, known as the speed of light. The time rates of changes of relevant electromagnetic fields are sufficiently low that they can be simplified to the set of quasi-static Maxwell equations\,\cite{Melcher1981ContinuumElectromechanics}. The quasi-static Maxwell equations in differential forms are defined as:
    
    
        \begin{equation}
            \nabla \times H = J_f
            \label{eq:ampere's circuit law}
        \end{equation}
        
        \begin{equation}
            \nabla \cdot B = 0
            \label{eq:gauss's magnetism law}
        \end{equation}
        
        \begin{equation}
            \nabla \times E = \frac{\partial B}{\partial t}
            \label{eq:maxwell-faraday's law}
        \end{equation}
    
        \begin{equation}
            \nabla \cdot D = \rho
            \label{eq:gauss's law}
        \end{equation}
    
    
        where $J_f$ is the free current density in conductor, $E$ is the electric field, $D$ is the electric flux density $\rho$ is the free electrical charge destiny. Bold symbol here represent vector quantities. Even existing in a simplified form, these equations are coupled \acsp{PDE}, which are often very difficult to solve analytically. Like any differential equation, boundary conditions and initial conditions are necessary for unique solution. Numerical methods can be used to approximate the result but exact solution is not yet found.
        
        
        The Maxwell equation can be rewritten to by introducing the magnetic vector potential $A$, true for quasi-static conditions:
        
        
        \begin{equation}
            \nabla \times A = B
            \label{eq:curl of A is B}
        \end{equation}        
        
        \begin{equation}
            M = M_0 + M_f
            \label{eq:component of magnetization}
        \end{equation}     
        
        \begin{equation}
            {\nabla}^2 A = -{\mu}_0 ({\mu}_r J_f + \nabla \times M_0)
            \label{eq:relation of A to Magnetization and free current}
        \end{equation}        
        
        
        The magnetization $M$ of a ferromagnetic material has two components. The first magnetization component $M_0$ result from permanent magnetization in hard magnetic material like permanent magnets. The secondary magnetization $M_s$ is related to the magnetic field $H$, which can be temporary:
        
        
        \begin{equation}
            M_s = \chi_m H
            \label{eq:secondary magnetization}
        \end{equation}     
        
        
        where $\chi_m$ is the magnetic susceptibly that can be a function of $H$ in non-linear materials. In a special case that there is no free current density, the magnetic scalar potential $\varphi$ is defined as:
        
        
        \begin{equation}
            - \nabla \varphi = H
            \label{eq:magnetic scalar and vector potential}
        \end{equation}     
        
        \begin{equation}
            \nabla^2 \varphi = \frac{1}{\mu_r} \nabla \cdot M_0
            \label{eq:magnetic scalar magnetization}
        \end{equation}     
        
        
        Note that Equation \ref{eq:relation of A to Magnetization and free current} and Equation \ref{eq:magnetic scalar magnetization} takes the form of the Poisson equation. The advantage of scalar potential is the reduced complexity compared the vector potential, which requires solving three-dimensional vector equations.
        
        
    % -----------------------------------------------------------------------------------
    % --- NEW SUB SECTION --- NEW SUB SECTION --- NEW SUB SECTION --- NEW SUB SECTION --- 
    % -----------------------------------------------------------------------------------
    \subsection{Ferromagnetic materials}                \label{Chapter:background/electromagnetic field theory/ferromagnetic material theory}
    
    
        The microscopic local magnetic alignment via dipoles spinning in ferromagnetic materials leads to the formation of magnetic ‘domains’. There are ‘hard’ ferromagnetic materials, and it is rather difficult to magnetize or demagnetize them. The reason is that the intrinsic material properties strongly resist the movement of domains wall. On the other hand, ‘soft’ ferromagnetic materials are easy to magnetize or demagnetize because the domain wall movement does not require a strong external $H$-field. The most common example of ‘soft’ magnetic behavior is when a piece iron is attracted to a permanent magnet, iron piece itself becomes temporarily magnetized to the extent that it can attract other iron pieces. 
        
        
        In case that the material exhibits no hysteresis, magnetization $M$ provides a measure for material response when there is a magnetic field $H$ applied to it. A general mathematical expression for the relationship between $B$-field, $H$-field and magnetization $M$ is shown below:
    
    
        \begin{equation}
            B = \mu_0 (H + M)
            \label{eq:magnetic field, field density and magnetization}
        \end{equation}    
    
    
        where $\mu_0$ is the permeability of free space, $\mathrm{4\pi \times 10^{-7} (T\cdot m)/A}$. In free space or weak magnetic material, magnetization $M$ is very weak, to such small magnitudes that it can be neglected. In hard ferromagnetic material, magnetization $M$ is very large compare to $H$-field. Thus, it is more convenient to characterize the materials by permeability $\mu$ and relative permeability $\mu_r$:
    
    
        \begin{equation}
            B = \mu H
            \label{eq:magnetic field and magnetic field density}
        \end{equation}   
        
        \begin{equation}
            \mu_r = 1 + \frac{M}{H} 
            \label{eq:realtive permeability definition}
        \end{equation}   
        
        \begin{equation}
            \mu = \mu_0 \mu_r 
            \label{eq:permeability definition}
        \end{equation}  
    
    
        The magnetization curve, also called the B-H curve defines two key parameters that describe the strength of permanent magnets: remanence $B_{rem}$ is the value of $B$-field when there is no applied $H$-field; and coercivity $H_{ci}$ is the value of $H$-field at which the value of $B$-field is zero. If the $H$-field is brought down to lower than $H_{ci}$, the magnetization is would never return to its original value. A high value of $B_{rem}$ is desirable in permanent magnets and magnetic memory components because the material would retain a large magnetization when external field is remove. The value of $H_{ci}$ determines how well does the material retain its magnetic property under the influence of an external and opposing H-field. Both $B_{rem}$ and $H_{ci}$ drop as temperature increases.
        
        
        \begin{figure*}[!ht]
            \centering
            \subfloat[]{
                \includegraphics[width=0.40\textwidth]{chap2/images/large_bh_loop.png}
                \label{fig:chapter/background/bh loop/large}
            }
            \qquad
            \subfloat[]{
                \includegraphics[width=0.4\textwidth]{chap2/images/narrow_bh_loop.png}
                \label{fig:chapter/background/bh loop/narrow}
            }
            \caption{
                Illustration of magnetic hysteresis for different ferromagnetic materials. Large $B-H$ curve (a) retains a significant fraction of magnetic flux density $B$ when the applied magnetic field $H$ is removed. Narrow $B-H$ curve (b) implies small energy dissipated when the applied magnetic field H is changing in the material.
            }   \label{fig:chapter/background/bh loop}
        \end{figure*}
        
        
        The gradient of $B-H$ curve is the value of material permeability $\mu$. Ferromagnetic material with linear $B-H$ relationship, or in other words, materials having the constant permeability over a range of applied H-field are relatively easier to model because losses caused by changing of H-field can be predicted without the need of iterative numerical methods. There has been many successes with modelling Neodymium NdFeB rare-earth permanent magnet because this material has high $B_{rem}$, high $H_{ci}$, and $B-H$ linearity over a wide range of applied $H$-field. Iron, however, has a relatively low $B_{rem}$ and narrow range of $B-H$ curve linearity which makes it difficult to predict losses caused by magnetic hysteresis. Although a lookup table can be created, however, such data table needs to include frequency in addition to the fields amplitude. 
        
        
        While a wide hysteresis loop, as illustrated in Figure\,\ref{fig:chapter/background/bh loop/large} may seem ideal for permanent magnets because the material will likely retain a large fraction of magnetization field once the driving field is removed. However, for materials used as a magnetic conductor to close the effective air gap like iron, a narrow hysteresis loop is desired. Narrow hysteresis loop, as shown in Figure\,\ref{fig:chapter/background/bh loop/narrow} implies that only a small amount of energy is lost during repeated shifting of the magnetization which occurs every time the magnet moves relative to the iron. Hysteresis loss is still a tough effect to solve analytically.
        
        
        As mentioned above, in magnetized ‘hard’ ferromagnetic material such as permanent magnet, there exists a net magnetic moment without an applied external field $H$. The more convenient representation for permanent magnet introduces $M_0$ as an additional magnetic moment term:


        \begin{equation}
            B = \mu_0 (H + M + M_0)
            \label{eq:B field equation 1}
        \end{equation}   
        
        \begin{equation}
            B = \mu_0 \mu_r H + \mu_0 M_0
            \label{eq:B field equation 2}
        \end{equation}   
        
        \begin{equation}
            M_0 = \frac{B_{rem}}{\mu_0}
            \label{eq:B rem equation}
        \end{equation}  
        
    
    % -----------------------------------------------------------------------------------
    % --- NEW SUB SECTION --- NEW SUB SECTION --- NEW SUB SECTION --- NEW SUB SECTION --- 
    % -----------------------------------------------------------------------------------
    \subsection{Force production}                   \label{Chapter:background/electromagnetic field theory/force production}
    
        Lorentz's force equation describe force $\overrightarrow{f}$ experienced by a charge $q$ moving with a velocity $\overrightarrow{v}$:
        
        \begin{equation}
            \overrightarrow{f} = q(\overrightarrow{E} + \overrightarrow{v} \times \overrightarrow{B})
            \label{eq:lorentz force equation}
        \end{equation}   
        
        where there is a electric field $\overrightarrow{E}$ and magnetic field $\overrightarrow{B}$. The first part of the equation describes the force exerted on a charge by mere presence of an electric field $\overrightarrow{E}$. The second part of the equation describe the force created when a moving charge is crossing a magnetic field $\overrightarrow{B}$. Force $\overrightarrow{F}$ is the resultant force acting on a body given volume $V$ with current density $J$ flowing through volume $V$:
        
      
        \begin{equation}
            \overrightarrow{F} = \int_V \overrightarrow{J} \times \overrightarrow{B} \mathrm{d}v
            \label{eq:volume generalized lorentz}
        \end{equation} 
        
        
        Instead of calculating the cross product of the free current density and the magnetic flux density, the total current density in the domain has to be considered, i.e., the sum of all free and microscopic currents at atomic level. The Maxwell stress tensor $\sigma$ description can derived on a body enclosed by surface area $S$:
        
        
        \begin{equation}
            \overrightarrow{F} = \oint_{S} \sigma \cdot \overrightarrow{n} \mathrm{d}s
            \label{eq:surface generalized lorentz}
        \end{equation}     

        where the Maxwell stress tensor, $\sigma$, consists of only the magnetic field component but not the electric field component, is coordinate system independent and defined as:

        \begin{equation}
            \label{eq:maxwell stress tensor}
            \sigma_{ij} = \frac{B_i B_j}{\mu_0} - \delta_{ij} \frac{| \overrightarrow{B} |^{2}}{2\mu_0}
        \end{equation}
                
        \begin{equation}
            \notag
            \begin{split}
                \text{where Krocher delta} \quad \delta_{ij} =
                    \begin{cases}
                        1       & \text{if $i=j$,}\\
                        0       & \text{if otherwise}
                    \end{cases}
            \end{split}
        \end{equation}


% ===================================================================================================
% === NEW SECTION === NEW SECTION === NEW SECTION === NEW SECTION === NEW SECTION === NEW SECTION ===
% ===================================================================================================
\section{Modelling techniques for motors}           \label{Chapter:background/modelling techniques for designing motors}
    \subsection{Numerical methods}                  \label{Chapter:background/modelling techniques for designing motors/numerical methods}
    \subsection{Analytical methods}                 \label{Chapter:background/modelling techniques for designing motors/analytical methods}
    \subsection{Semi-analytical methods}            \label{Chapter:background/modelling techniques for designing motors/semi-analytical methods}
    \subsection{Empirical methods}                  \label{Chapter:background/modelling techniques for designing motors/empirical methods}


% ===================================================================================================
% === NEW SECTION === NEW SECTION === NEW SECTION === NEW SECTION === NEW SECTION === NEW SECTION ===
% ===================================================================================================
\section{Optimization methods}                      \label{Chapter:background/optimization methods}
    \subsection{Classification}                     \label{Chapter:background/optimization methods/classification}
    \subsection{Application in motor design}        \label{Chapter:background/optimization methods/application in motor design}


% ===================================================================================================
% === NEW SECTION === NEW SECTION === NEW SECTION === NEW SECTION === NEW SECTION === NEW SECTION ===
% ===================================================================================================
\section{Summary}                                   \label{Chapter:background/summary}